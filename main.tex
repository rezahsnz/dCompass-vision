\documentclass[12pt, a4paper]{article}
\usepackage[utf8]{inputenc}
\usepackage{hyperref}
\usepackage[
backend=biber,
style=alphabetic,
sorting=ynt]{biblatex}
\addbibresource{references.bib} 

\title{dCompass, An Interactive Web3 Platform For Learning}
\author{Reza Hasanzadeh \\ \href{mailto:rezahsnz@protonmail.com}{rezahsnz@protonmail.com} \\ Draft}
\date{April, 2022}

\begin{document}

\maketitle

\begin{abstract}
This essay describes a web3 learning platform whose goal is to create an interactive educational platform through the use of mnemonic memory models. Content creators prepare a set of cards and embed them within educational material. To boost users’ memory, each card is presented in customizable repeated test and review sessions. Users' evaluation is conducted with the help of a system of smart contracts. Achievement proofs are represented by the issuance of non-fungible tokens. The learning and evaluation process are designed in way to ensure that the gamified approaches keep users incentivized to learn about projects, undertake new learning pathways, try quests, collect badges, form guilds, and other common activities.
\end{abstract}

\section*{Introduction}
In retrospect Papert was right, "One of the best ways to learn French is to spend one week or two on a vacation in France."\cite{papert}. So what makes the natural environments so good at teaching? The answer may lie in the presence of such an immense amount of the rich feedback loops. Imagine a kid playing with mud on the beach: the feeling of the moist sand on the hands, the breeze blowing from the sea, the water waves and how they collide with each other and with the rocks, dolphins jumping in and out of water, surfers riding wild waves, the sound of the birds flying around, the graceful dance of tiny crabs, and finally the muddy castle that's being completed, all are united to provide a rich environment to stimulate the kid's mind. So, the ultimate goal of any learning platform should be creating an environment as rich as the natural world. This goal may be impossible to achieve but approximations would work well too. With powerful personal computers being around for the past 30 years, there have been many attempts to bring that level of richness available in natural environments. Among all attempts, video games seem to be the most successful example of learning environments and with “successful” we do mean not mean financially but rather qualitatively. The quality of learning in video games is so high that sometimes it is hard to tell if it is a game or the reality. The reason of this success is the simple fact that video games try their best to simulate or sometimes surpass the reality. A video game is usually comprised of two distinct phases: a demo learning session, and the real action. The learning session provides the user with all the important information needed to succeed. From honing her various skills to the introduction of different items, places, enemies, and challenges she is going to encounter, all in a compelling fashion. Then comes the real action phase where the player is placed in a high-quality simulation of the real or some fantasy world. In a first-person shooter game, for example, the user can kill, die, and get injured as if she is in a real battle with some compromises though. Play to learn is a great way to teach any topic to humans and video games seem to follow this narrative. When the game is finally finished, the player has learned all the necessary topics she was assumed to learn by the end of the game even though she overlooked the fact that she was actually playing. Video games, by truly harnessing the power of computers, blend reality into fantasy in a way that is hardly seen in other inventions of humans.

If we assume that computers are tools capable of amplifying our intellectuals, and they truly are, then we have to give them the content they deserve. This content is simply interactive and dynamic processes that are impossible to have on any other medium. Well, when we leave the entertainment sector and consider consumer oriented products, like word processors for example, we observe that almost all of these products barely go beyond what is a simple static processes. They literally simulate paper on computers as if it is 17 century. The situation gets worse in educational tools where the  burden of learning is completely put on the user’s shoulders, "here is the (arcane) notation and symbols we use, you need to master it, print it, read and come back for evaluation!". This is fine if done on paper but when it comes to the computers and their revolutionary capabilities, it is completely disheartening to see such expectations. So, we think that if the content is going to be consumed on a computer, then there should be some interactive environment enriched with dynamic elements whose state variables are controlled by the user. Such setups welcome the exploratory nature of humans and let them to learn by playing as if they were doing it in some reality world. For example, imagine that Uniswap is going to educate people on how its AMM system works. There is a certain amount of math involved and the notation should be mastered ultimately. So, what is the best way to teach users here? How about pairing the mathematical content with an interactive widget that allows users to modify any state variable of the pool while watching in real-time how the pool reacts to those changes? Well, this scheme seems pretty natural and wanting for users with one immediate outcome: users would never want to go back to the era of paper simulation. This setup enables humans to have virtual labs about any scientific topic and lets them to modify its state variables and track its evolution over in real-time, a mini-Mathland\cite{papert}.

Any learning platform shall facilitate means for evaluation too. Once users have played enough with  educational content, they would look for ways to evaluate their knowledge either out of interest or some proof for an external entity. One should note that if the learning phase happened in a rich environment, the evaluation phase too needs to be conducted in a environment whose settings resemble that of the learning phase. To maintain evaluation results, the blockchain technology provides an open and secure way. The idea is to employ NFTs and smart contracts for evaluation proofs. If any user passed a test, a set of non-fungible tokens would be minted for her as the proof of her success. Entities can build upon this and have their open evaluation tests. Uniswap, for example, can ask potential hires to "prove" that they have passed the required tests, therefore, automating away some of the on-boarding steps found in pre-blockchain systems. This on-chain representation of achievements paves the way for various forms of compositions: users can be assigned different roles within communities based on their achievements, they can form various groups, there could be different types and sequences of competitions, challenges, quests, rankings, and ... all in a verifiable way.

dCompass is a decentralized learning platform that tries to combine the two processes mentioned above with the aim of creating a secure, open, and humane learning environment.

\section*{The Interface}
Mainstream learning platforms give users some educational material, usually in PDF, web, ... format or video clips if they are really generous, and later evaluate them based on that. This semi-unsupervised practice has been around for at least 500 years and seems to be very well defined and agreed upon by everyone as a low cost solution, well, the cost is actually transferred to the learner. While this approach has certain benefits, we argue that it is obsolete in the era of computers. Cognitive and learning researchers have made important progress in the past 50 years about how humans learn. Of special interest are the methods that automate away the memorization of topics. Schemes like space repeated memory, elaboartive encoding, and mnemonic memory models have shown promising results for extended periods of retains. We think that the time is ripe to use these results in our everyday computing tools.

Learning seems to involve heavy amounts of memorization. Of all the concepts usually confronted in  an educational material, mathematical ones seems to be the hardest to learn. Every scientific or technological concept has some mathematical notations and symbols to master and it appears that memorizing initial notations and symbols, which are often simple, help a lot for later mastery of more advanced concepts\cite{nielsen}. In recent years, some tools have started to incorporate memorizing elements in their interface. One of them, Quantum Country, teaches quantum computing concepts through expert-tailored test and review cards. Quantum Country builds upon the idea of mnemonic memory models\cite{matuschak}. The idea is to embed test and review cards right within the educational material. The process is usually as follows: once a section is read, the user is tested immediately and if she passes the test, meaning that she knows the subject being discussed, she will be tested some later time. This test and review process continues until the system has made sure that the content is internalized. The time between review tests grows exponentially, an example could be a)immediately, b)one day later, c)three days later, d)one week later, and e)two months later. If a review card is failed, the next review session happens in the opposite direction, e.g. if c failed, the next review session is going to be b. It is argued that this sequential test and review scheme, while being brief and low demanding, boosts memory retention of concepts for up to 10 years and more\cite{nielsen}\cite{matuschak}.
dCompass extends the mnemonic memory model of Quantum Country by introducing interactive cards.  An interactive card could be as bare as a piece of common static text, to as simple as having some sliders that control a blockchain's state variables, and to as complex as a special script that initiates a black swan event on markets to observe the evolution of AMMs in response to that event. By this hierarchy of dynamism we want to make sure that every need is covered even though the preferred choice is always a rich environment. This design pulls some of the educational burden off learners and distributes the costs more evenly between the creators and the learners. Material-wise, we have dynamic educational material as well as dynamic test and review cards. The success of any material is tightly tied to how good it provides its content, the more dynamic the more wanting. Any material created within dCompass, subject to its creator’s consent, is extendable. So, forking and building on top of other works is available, this is achieved through a marketplace. The dynamic elements within any material shall be addressable too so that creators are able to include other works in their own. The review times for cards should be chosen by the creators. However, this does not mean that it is an static property of the material and learners have the freedom to change the times as they wish, this is due to the fact that such hard-coded settings does not work for everyone the same. Now that we have a conceptual framework of dCompass's aspirations, let’s see how such an interface would look and feel in cyberspace.

At the top level we have the notion of projects. A project is an umbrella term for everything that happens within the application. Each project has a creator and the proof of the creation is written to the blockchain in the form of NFT mints. A project can be sponsored and currently there are three different types of sponsor passes: "Bronze", "Silver", and "Diamond". Each sponsor pass enables certain degrees of controls over the project. Within each project there are be different pathways for users to undertake. Think of a pathway as a sequence of steps required to learn a topic and pass certain evaluation tests. Each pathway is comprised of a series of tests called quests that are the actual evaluation phase of the learning process. A quest could have a limited number rewards acting as an incentive force for test takers. A quest can have other quests as prerequisites to make sure that the learning package is complete. Users are free to explore and try different pathways and quests. There exists recommendation engines that act as guided tours to the users suggesting the next best step. Once a quest is passed, the success is represented by a set of NFT mints. Just like a quest, once a pathway is completed, another set of tokens are minted to seal the achievement. The user who goes through pathways and takes quests is called an adventurer. An adventurer can expect to receive monetary rewards too subject to availability by the project. Moreover, an adventurer who passes certain quests or pathways receives special tokens called badges. These badges could further be used to create a skill-tree for adventurers. Adventurers can also form guilds and special groups based on their achievements with their earned tokens acting as membership tickets.

dCompass is designed in a way that every major event that happens is recorded on-chain and thus is verifiable. However, there are certain information that are not recorded on-chain. Bulky and meta-like information whose on-chain recording are not feasible, get persisted to decentralized off-chain storage platforms and receipts are linked to the on-chain tokens.

\renewcommand{\abstractname}{Conclusion}
\begin{abstract}
Equipped with right soft machinery, computers can literally act as powerful external brains to us and expand our understanding. Learning is one of those areas that could benefit a lot from computers. In this essay, we speculated about tools that we think could make it easier for humans to learn and memorize concepts for extended periods of time. While no panacea exists for the hard problem of education, as we progress promising new methods are found that are needed to be put to work. With learning comes the need for evaluation too. Programmable blockchains provide open and secure ways for this evaluation to work. Various NFTs can be minted and transferred as proofs for certain achievements. A humane learning platform coupled with the right set of smart contracts would automate away lengthy in-person paper-works usually dealt with in educational settings. dCompass is designed with those goals in mind in a way to be both open and secure while being stimulating and appealing to the end users.
\end{abstract}

\printbibliography

\end{document}